% Created 2015-12-08 Tue 09:51
\documentclass[11pt]{article}
\usepackage[utf8]{inputenc}
\usepackage[T1]{fontenc}
\usepackage{fixltx2e}
\usepackage{graphicx}
\usepackage{longtable}
\usepackage{float}
\usepackage{wrapfig}
\usepackage{rotating}
\usepackage[normalem]{ulem}
\usepackage{amsmath}
\usepackage{textcomp}
\usepackage{marvosym}
\usepackage{wasysym}
\usepackage{amssymb}
\usepackage{hyperref}
\tolerance=1000
\usepackage{fullpage}
\usepackage{times}
\date{2015-07-14}
\title{EPICS on Raspberry Pi}
\hypersetup{
  pdfkeywords={},
  pdfsubject={},
  pdfcreator={Emacs 24.3.1 (Org mode 8.2.4)}}
\begin{document}

\maketitle

\section{Install notes}
\label{sec-1}
Based on \url{http://prjemian.github.io/epicspi/}\\
  This assumes you have a clean install of raspbian (I'm using 2015-05-05)

\subsection{EPICS Base}
\label{sec-1-1}
\begin{itemize}
\item Boot into raspbian and make sure you have a network connections
\item Log in as user: pi
\item Set up the date (if it's wrong!)
\begin{verbatim}
sudo dpkg-reconfigure tzdata
\end{verbatim}
\item Make a new user: enge (make whatever you want for the password)
\begin{verbatim}
sudo adduser --home /home/enge enge
\end{verbatim}
\item Add that user to the sudoers list
\begin{verbatim}
sudo adduser enge sudo
\end{verbatim}
\item Log out and then in as that user
\item Make the file structure
\begin{verbatim}
mkdir bin
mkdir extensions
mkdir project
mkdir synApps
\end{verbatim}
\item Download and untar the EPICS base package
\begin{verbatim}
wget http://www.aps.anl.gov/epics/download/base/base-3.15.3.tar.gz 
tar -xzvf base-3.15.3.tar.gz
ln -s base-3.15.3 base
\end{verbatim}
\item Paste the following into .bashrc
\begin{verbatim}
######################################################################
##  EPICS
######################################################################

## Base
export EPICS_ROOT=/home/enge
export EPICS_BASE=${EPICS_ROOT}/base/
export EPICS_HOST_ARCH=`${EPICS_BASE}/startup/EpicsHostArch`
export EPICS_BASE_BIN=${EPICS_BASE}/bin/${EPICS_HOST_ARCH}
export EPICS_BASE_LIB=${EPICS_BASE}/lib/${EPICS_HOST_ARCH}
if [ "" = "${LD_LIBRARY_PATH}" ]; then
    export LD_LIBRARY_PATH=${EPICS_BASE_LIB}
else
    export LD_LIBRARY_PATH=${EPICS_BASE_LIB}:${LD_LIBRARY_PATH}
fi
export PATH=${PATH}:${EPICS_BASE_BIN}

## EPICS Extensions
export EPICS_EXT=${EPICS_ROOT}/extensions
export EPICS_EXT_BIN=${EPICS_EXT}/bin/${EPICS_HOST_ARCH}
export EPICS_EXT_LIB=${EPICS_EXT}/lib/${EPICS_HOST_ARCH}
if [ "" = "${LD_LIBRARY_PATH}" ]; then
    export LD_LIBRARY_PATH=${EPICS_EXT_LIB}
else
    export LD_LIBRARY_PATH=${LD_LIBRARY_PATH}:${EPICS_BASE_LIB}
fi
export EPICS_SYNAPPS_BASE=${EPICS_ROOT}/synApps
export EPICS_SYNAPPS_BIN=${EPICS_SYNAPPS_BASE}/support/utils
export PATH=${PATH}:${EPICS_EXT_BIN}:${EPICS_SYNAPPS_BIN}
\end{verbatim}
\item Load it
\begin{verbatim}
source ~/.bashrc
\end{verbatim}
\item Compile EPICS
\begin{verbatim}
cd base
make -j3
\end{verbatim}
\item Buy and drink some coffee
\item Once finished, check that it works
\begin{verbatim}
softIoc
\end{verbatim}
\item Epics should have started. Now run the IOC
\begin{verbatim}
iocInit
\end{verbatim}
\item You should see something that looks like
\begin{verbatim}
epics> iocInit  
Starting iocInit
############################################################################
## EPICS R3.15.3 $Date: Thu 2015-05-14 14:09:28 +0200$
## EPICS Base built Jul 17 2015
############################################################################
iocRun: All initialization complete
epics> exit
\end{verbatim}
\end{itemize}
\subsection{synApps (needed for streamdevice and serial connections)}
\label{sec-1-2}
\begin{itemize}
\item Get the extensions and msi first
\begin{verbatim}
cd ~
wget http://www.aps.anl.gov/epics/download/extensions/extensionsTop_20120904.tar.gz
tar -xzvf extensionsTop_20120904.tar.gz
wget http://www.aps.anl.gov/epics/download/extensions/msi1-7.tar.gz
cd extensions/src
tar -xzvf ../../msi1-7.tar.gz
cd msi1-7
make
\end{verbatim}
\item Install re2c (I don't know what it's for)
\begin{verbatim}
sudo apt-get install re2c
\end{verbatim}
\item Download and unzip synApps
\begin{verbatim}
cd ~
wget http://www.aps.anl.gov/bcda/synApps/tar/synApps_5_8.tar.gz
tar -xzvf synApps_5_8.tar.gz
ln -s synApps_5_8 synApps
\end{verbatim}
\item We don't need all the junk included
\begin{verbatim}
cd synApps/support/configure
emacs RELEASE
\end{verbatim}
\item Edit the SUPPORT line
\begin{verbatim}
SUPPORT=/home/enge/synApps/support
\end{verbatim}
\item Edit EPICS$_{\text{BASE}}$
\begin{verbatim}
EPICS_BASE=/home/enge/base
\end{verbatim}
\item Comment out (with a '\#') the modules we don't want
\begin{itemize}
\item \verb~ALLEN_BRADLEY~
\item \verb~AREA_DETECTOR~
\item \verb~ADCORE~
\item \verb~ADBINARIES~
\item \verb~CAPUTRECORDER~
\item \verb~CAMAC~
\item \verb~DAC128V~
\item \verb~DXP~
\item \verb~IPUNIDIG~
\item \verb~OPTICS~
\item \verb~QUADEM~
\item \verb~SOFTGLUE~
\item \verb~VME~
\end{itemize}
\item Prepare the makefile
\begin{verbatim}
cd ~/synApps/support
make release
\end{verbatim}
\item Compile!
\begin{verbatim}
make -j3 rebuild
\end{verbatim}
\end{itemize}
\subsection{Tidy up}
\label{sec-1-3}
\begin{itemize}
\item Make a folder to keep zip files
\begin{verbatim}
cd ~
mkdir download
mv *.tar.gz download
\end{verbatim}
\end{itemize}
% Emacs 24.3.1 (Org mode 8.2.4)
\end{document}